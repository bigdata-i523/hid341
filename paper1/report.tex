\documentclass[sigconf]{acmart}

\usepackage{hyperref}

\usepackage{endfloat}
\renewcommand{\efloatseparator}{\mbox{}} % no new page between figures

\usepackage{booktabs} % For formal tables

\settopmatter{printacmref=false} % Removes citation information below abstract
\renewcommand\footnotetextcopyrightpermission[1]{} % removes footnote with conference information in first column
\pagestyle{plain} % removes running headers

\begin{document}
\title{Big Data Applications for Clinical Trials}


\author{Jacob Tibenkana}
\orcid{1234-5678-9012}
\affiliation{%
  \institution{Indiana University}
  \streetaddress{107 S. Indiana Avenue}
  \city{Bloomington}
  \state{Indiana}
  \postcode{47405-7000}
}
\email{jtibenka@iu.edu)

% The default list of authors is too long for headers}
\renewcommand{\shortauthors}{B. Trovato et al.}


\begin{abstract}
This  paper will  provide a brief overview  of how Pharmaceutical companies use big data applications to track and analyze data for clinical trials. This paper will also explore the relationship between data quality and the intergrity of the clinical studies. Today, many trails are conducted across the world in many fields such as oncology, rare diseases, central nervous diseases, etc. These trails aim to further advance treatments for diseases in the these fields. Clinical trials have a long history. "The world's first clinical trial is recorded in the Book of Daniel in the Bible. During his rule in Babylon, Nebuchadnezzar ordered his people to eat only meat and drink only wine, but several young men of royal blood, who preferred to eat vegetables, objected. The king allowed these rebels to follow a diet of legumes and water — but only for 10 days. When Nebuchadnezzar's experiment ended, the vegetarians appeared better nourished than the meat-eaters." (Bhatt,pg.6,2010). Now this maybe a good example for a simple straight forward trial, however today's trails are more complicated and can last longer than 10 days, usaully a period of 12 months to 60 months. This is only the tip of the iceburg, because they is all sorts of cretials (inclusive and exclusive) that have to be adhered to for subject selection. The study design and documention have to be approved by the FDA and IRB, contrats between sites, CROs, sponsors, vendors, and licensors that have to be executed in order for the trial to proceed. Subject recrutment has to occur and raters have to be trained to conduct assessments with the subjects on a specified study schedule, etc. As one can see there are many factors to consider and much work that goes into running and completing a clinical trial today. Thankfully, with the advancement in technology, these factors are becoming easier and easier to manange, because all data management, tracking, and analysis.
\end{abstract}

\keywords{i523}


\maketitle

\section{Introduction}

The \textit{proceedings} are the records of a
conference. ACM seeks to give these
conference by-products a uniform, high-quality appearance.  To do
this, ACM has some rigid requirements for the format of the
proceedings documents: there is a specified format (balanced double
columns), a specified set of fonts (Arial or Helvetica and Times
Roman) in certain specified sizes, a specified live area, centered on
the page, specified size of margins, specified column width and gutter
size \cite{editor00}.


\begin{acks}

  The authors would like to thank 

\end{acks}

\bibliographystyle{ACM-Reference-Format}
\bibliography{report} 

\end{document}
