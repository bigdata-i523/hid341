\documentclass[sigconf]{acmart}

\usepackage{hyperref}

\usepackage{endfloat}
\renewcommand{\efloatseparator}{\mbox{}} % no new page between figures

\usepackage{booktabs} % For formal tables

\settopmatter{printacmref=false} % Removes citation information below abstract
\renewcommand\footnotetextcopyrightpermission[1]{} % removes footnote with conference information in first column
\pagestyle{plain} % removes running headers

\begin{document}
\title{Big Data Applications for Clinical Trials}


\author{Jacob Tibenkana}
\orcid{1234-5678-9012}
\affiliation{%
  \institution{Indiana University}
  \streetaddress{107 S. Indiana Avenue}
  \city{Bloomington}
  \state{Indiana}
  \postcode{47405-7000}
}
\email{jtibenka@iu.edu)

% The default list of authors is too long for headers}
\renewcommand{\shortauthors}{B. Trovato et al.}


\begin{abstract}
This  paper will  provide a brief overview  of how Pharmaceutical companies use big data applications to track and analyze data 
for clinical trials. This paper will also explore the relationship between data quality and the intergrity of the clinical studies. 
\end{abstract}

\keywords{i523}


\maketitle

%here begins the body of the document
\section{Introduction}
Today, many trails are conducted across the world in many fields such as oncology, rare diseases, central nervous diseases, etc. 
These trails aim to further advance treatments for diseases in the these fields. 
Clinical trials have a long history. "The world's first clinical trial is recorded in the Book of Daniel in the Bible. During his rule in Babylon, Nebuchadnezzar ordered his people to eat only meat and drink only wine, but several young men of royal blood, who preferred to eat vegetables, objected. The king allowed these rebels to follow a diet of legumes and water — but only for 10 days. When Nebuchadnezzar's experiment ended, the vegetarians appeared better nourished than the meat-eaters." (Bhatt,pg.6,2010). 
Now this maybe a good example for a simple straight forward trial, however today's trails are more complicated and can last longer than 10 days, usaully a period of 12 months to 60 months. 
This is only the tip of the iceburg, because they is all sorts of cretials (inclusive and exclusive) that have to be adhered to for subject selection. The study design and documention have to be approved by the FDA and IRB, contrats between sites, CROs, sponsors, vendors, and licensors that have to be executed in order for the trial to proceed. 
Subject recrutment has to occur and raters have to be trained to conduct assessments with the subjects on a specified study schedule, etc. As one can see there are many factors to consider and much work that goes into running and completing a clinical trial today. Thankfully, with the advancement in technology, these factors are becoming easier to manange, because af data management, from a perspective of data tracking and analysis using, TMF, EDC, and other vendor databases/services.

\section{TMF, CTMs, EDC, and Vendor systems}
If an entity (person, company, organization, etc) plans to conduct a clinical trial, how do they go about addressing the
factors mentioned above. Lets exclude the labor part such as the day to day tasks, and deliverables of the clinical study, but think of each factor above as various information that can be collected, tracked, and most importantly stored. 
This is where the use of big Data applications and analysis comes into the play. 
According to the National Insistute for Health Research (NHS), a trial master file (TMF) should be made at the intial phase of the trial. "TMF is the collection of documentation that allows the conduct of the clinical trial, the integrity of the trial data and the compliance of the trial with GCP to be evaluated.
It is also essential to allow the trial to be effectively managed by the sponsor as it allows the appropriate individuals access to the necessary trial documentation." (Pearce, Apr 18, 2016) 
There is a large amount of documentation that must be collected, stored (for audit puropses), and submitted to the government for approval for a trial to happen. 
Thanks to the big data applications field, electronic TMFs (eTMF) have been created to host all these documents for the sponsor and CROs. This information can be accessed and retrived for various purposes depending on which phase the study is in (start up, mid, data lock, and close out). 
Some of these eTMFs are through MasterControl, "a document control solution provider" (MasterControl, 2000). Other eTMs are in-house platforms provided by vendor companies for the trials. 
Some of these vendors companies include Phlex Global, SureClinical, etc.
Another area that falls into the big data applications aspect for clinical trials is the software Clinical Trial Management System (CTMS).
According to an article called CTMS: What You Should Know, sponsors use CTMS to exceed organizational limits, preparation, reporting, timelines, be diligent, manage distribution of drug supplies, tracking, and offer data to business intelligency systems that are used as digital dashboards for clinical trail managers. (Dagalur, Mar 16, 2016) 
Sponsors can have their own CTMS or typically they will contract other companies (for eaxmple, Simpletrials, Allegro, Parexel, etc) with their own in-house CTMS to manage this aspect of the trial. 
Before computers, one would assume that subject data from trails was recorded on paper, however today, databases play a huge role. For example, the electronic data capture.
According to Kristina Lopienski artilce, The Beginner’s Guide to an Electronic Data Capture (EDC) System, the EDC is a database that stores subject dat collected in a clinical study. 
More and more studies are using EDC software and replacing traditional paper records. 
The article goes on to say that different EDC systems are used by sponsors, CROs, sites, and vendors to manage data in the aspects of tracking, quick access, security, and oganization, and compliance. (Lopienski,February 2, 2016)


The \textit{proceedings} are the records of a
conference. ACM seeks to give these
conference by-products a uniform, high-quality appearance.  To do
this, ACM has some rigid requirements for the format of the
proceedings documents: there is a specified format (balanced double
columns), a specified set of fonts (Arial or Helvetica and Times
Roman) in certain specified sizes, a specified live area, centered on
the page, specified size of margins, specified column width and gutter
size \cite{editor00}.


\begin{acks}

  The authors would like to thank 

\end{acks}

\bibliographystyle{ACM-Reference-Format}
\bibliography{report} 

\end{document}
